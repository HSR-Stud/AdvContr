\section{G"ute}
\subsection{LQR-Entwurf (Linear-quadratic regulator)}
	$$I(K_R,T_N,\ldots)=\int\limits^{\infty}_0 e(t)^2 dt = minimal$$\\
	$$I(k)=\int\limits_0^{\infty} {x^T Q x+u^T R u}= minimal$$\\
	Q: symmetrische positive(oft Diagonalmatrix); Bei SISO ist R eine Zahl\\
	Optimaler Regler ist dann ein Riccati-Regler:\\
	\begin{minipage}{10cm}
	allgemeine Darstellung (n-te Ordnung):\\
	$$ -A^T S - S A + S b R^{-1} b^T S - Q = 0$$
	\end{minipage}
	\begin{minipage}{6cm}
    1.Ordnung:\\
    $$-2aS+\frac{b^2S^2}{r}-q=0$$
    \end{minipage}\\
	S ist die L"osungsmatrix n-ter Ordnung 
	\begin{aufzaehlung}
    	\item Vorgabe der symmetrischen positiven Bewertungsmatrix Q und R
    	\textbf{meist} wird f"ur Q=I und R=1 eingesetzt.
    	\item L"osungsmatrix S bestimmen (aus der L"osungsmenge den positiv
    	definierten Wert)
    	\item Reglervektor berechnen $k^T=R^{-1} b^T S$
    \end{aufzaehlung}
