% Genereller Header
\documentclass[10pt,twoside,a4paper,fleqn]{article}
\usepackage[utf8]{inputenc}
\usepackage[left=1cm,right=1cm,top=1cm,bottom=1cm,includeheadfoot]{geometry}
\usepackage[ngerman]{babel,varioref}

% Pakete
\usepackage{amssymb,amsmath,fancybox,graphicx,color,lastpage,wrapfig,fancyhdr,hyperref,verbatim,floatflt,multirow,rotating,tabularx,enumitem,subfig,siunitx,pdfpages}


%%%%%%%%%%%%%%%%%%%%
% Generelle Makros %
%%%%%%%%%%%%%%%%%%%%
\newcommand{\formelbuch}[1]{\footnotesize{(Skript S. #1)}\normalsize{}}
\newcommand{\verweis}[2]{\small{(siehe auch \ref{#1}, #2 (S. \pageref{#1}))}}
\newcommand{\subsubadd}[1]{\textcolor{black}{\mbox{#1}}}

\newcommand{\matlab}[1]{\footnotesize{(Matlab: \texttt{#1})}\normalsize{}}


\newenvironment{liste}[0]{
	\begin{list}{$\bullet$}{ \leftmargin=0.5cm \setlength{\itemsep}{0cm}\setlength{\parsep}{0cm} \setlength{\topsep}{0cm}}}
  {\end{list}}
\newenvironment{aufzaehlung}[0]{
    \begin{enumerate}\setlength{\leftmargin=0.5cm \itemsep}{1pt}\setlength{\parskip}{0pt}\setlength{\parsep}{0pt}}
  {\end{enumerate}}      

\newcommand{\abbHeight}[3]{
	\begin{center}
		\includegraphics[height=#2]{./bilder/#1} \\
		#3
    \end{center}
}

%%%%%%%%%%
% Farben %
%%%%%%%%%%
\definecolor{black}{rgb}{0,0,0}
\definecolor{red}{rgb}{1,0,0}
\definecolor{white}{rgb}{1,1,1}
\definecolor{grey}{rgb}{0.8,0.8,0.8}

%%%%%%%%%%%%%%%%%%%%%%%%%%%%
% Mathematische Operatoren %
%%%%%%%%%%%%%%%%%%%%%%%%%%%%
\DeclareMathOperator{\sinc}{sinc}



% Fouriertransformationen
\unitlength1cm
\newcommand{\FT}
{
\begin{picture}(1,0.5)
\put(0.2,0.1){\circle{0.14}}\put(0.27,0.1){\line(1,0){0.5}}\put(0.77,0.1){\circle*{0.14}}
\end{picture}
}
\newcommand{\DFT}
{
\overset{DFT}{
	\begin{picture}(1,0.2)
	\put(0.2,0.1){\circle{0.14}}{\put(0.27,0.1){\line(1,0){0.5}}}\put(0.77,0.1){\circle*{0.14}}
	\end{picture}
}
}


\newcommand{\IFT}
{
\begin{picture}(1,0.5)
\put(0.2,0.1){\circle*{0.14}}\put(0.27,0.1){\line(1,0){0.45}}\put(0.77,0.1){\circle{0.14}}
\end{picture}
}

\newcommand{\IDFT}
{
\overset{IDFT}{
    \begin{picture}(1,0.2)
	\put(0.2,0.1){\circle*{0.14}}\put(0.27,0.1){\line(1,0){0.45}}\put(0.77,0.1){\circle{0.14}}
	\end{picture}
}
}



%%%%%%%%%%%%%%%%%%%%%%%%%%%%
% Allgemeine Einstellungen %
%%%%%%%%%%%%%%%%%%%%%%%%%%%%
%pdf info
\hypersetup{pdfauthor={\authorinfo},pdftitle={\titleinfo},colorlinks=false}
\author{\authorinfo}
\title{\titleinfo}

%Kopf- und Fusszeile
\pagestyle{fancy}
\fancyhf{}
%Linien oben und unten
\renewcommand{\headrulewidth}{0.5pt} 
\renewcommand{\footrulewidth}{0.5pt}

\fancyhead[L]{\titleinfo{ }\tiny{(\versioninfo)}}
%Kopfzeile rechts bzw. aussen
\fancyhead[R]{Seite \thepage { }von \pageref{LastPage}}
%Fusszeile links bzw. innen
\fancyfoot[L]{\footnotesize{\authorinfo}}
%Fusszeile rechts bzw. ausen
\fancyfoot[R]{\footnotesize{\today}}

% Einrücken verhindern versuchen
\setlength{\parindent}{0pt}


%%============Aufzählungen===========%%
%%=======\usepackage{enumitem}=======%%
\setlist{itemsep=-2pt}
\setlist{parsep=-2pt}
